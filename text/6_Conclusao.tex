\chapter{Considerações Finais} \label{cap:Considerações Finais}

Este trabalho demonstrou como tecnologias digitais inovadoras podem ser aplicadas à preservação e divulgação do patrimônio arqueológico, especificamente das pinturas rupestres do sítio Lapa da Pedra, em Formosa-GO. 
Cumpriram-se os objetivos do trabalho ao desenvolver uma plataforma digital que combina um site moderno e um ambiente virtual imersivo para a Lapa da Pedra. Conforme demonstrado nos Resultados (Seção \ref{cap:resultados}). O novo site atingiu ótimas métricas em usabilidade, desempenho e acessibilidade e o ambiente virtual permitiu uma exploração detalhada das pinturas rupestres.
A implementação deste trabalho não apenas cumpriu o objetivo proposto, mas também abriu portas para novas formas de educação patrimonial.

\section{Limitações e Dificuldades}
\label{sec:limitações}
Durante o desenvolvimento, algumas dificuldades foram enfrentadas, como por exemplo:
\begin{itemize}
    \item Dificuldades na otimização dos modelos 3D para uso na Unreal Engine;
    \item Limitação de tempo para implementar todas as funcionalidades desejadas;
    \item Restrições no orçamento;
    \item Limitações hardware para produzir os modelos 3D e o ambiente virtual na Unreal. Houveram processos que demoraram mais de 30 horas para renderizar;
    \item Compatibilidade: O ambiente virtual não foi testado em hardware muito antigo, restringindo o acesso a usuários com computadores básicos.
\end{itemize}

\section{Trabalhos Futuros} \label{sec:futuro}

Para expandir o projeto, diversas melhorias e novas funcionalidades podem ser implementadas. Uma sugestão é a integração com Realidade Virtual (VR), adicionando suporte para headsets como \textit{Oculus Rift} ou \textit{Meta Quest}, o que permitiria uma imersão total no ambiente virtual, proporcionando uma experiência mais envolvente para os usuários. Além disso, seria interessante desenvolver uma versão mobile do projeto, com um aplicativo para Android e iOS que ofereça recursos simplificados de visualização 3D, tornando o conteúdo acessível a um público ainda maior.
Outra possibilidade seria incorporar modelos 3D de outros sítios arqueológicos, utilizando a mesma infraestrutura já desenvolvida. Também se pode considerar a implementação de um chatbot com inteligência artificial no site, capaz de interagir com os visitantes e fornecer informações detalhadas sobre os sítios arqueológicos e o patrimônio cultural.


\subsection{Contribuição para a Área}
Este trabalho contribuiu para a interseção entre Arqueologia e tecnologia ao oferecer um modelo replicável para digitalização de sítios arqueológicos em regiões subrepresentadas e ao promover a conscientização sobre a importância da preservação cultural através de ferramentas acessíveis.

\subsection{Palavras Finais}
A preservação do patrimônio arqueológico é um desafio árduo e contínuo. Espera-se que esta iniciativa inspire novas pesquisas e políticas públicas voltadas à proteção digital de nossa herança cultural. 