This undergraduate thesis presents the development of a digital platform aimed at preserving and promoting the archaeological heritage of the Formosa region in Goiás, Brazil. The project involves the creation of a modern, responsive website combined with an immersive three-dimensional virtual environment that enables users to explore the rock paintings of the Lapa da Pedra archaeological site. The methodology employed was based on Design Science Research (DSR), structured in iterative development cycles. Photogrammetry and 3D modeling techniques were integrated into Unreal Engine 5.4 to create the virtual environment, while the website was developed using modern technologies such as Next.js and Sanity CMS. The results demonstrate a significant advancement in the digital exploration and preservation of archaeological heritage through the virtual environment, along with substantial improvements over previous presentation methods, fully meeting usability, accessibility, and performance requirements. This project contributes to heritage education and the digital preservation of cultural heritage, serving as a replicable model for other archaeological sites.

\begin{keywords}
Digital Archaeology, Photogrammetry, Lapa da Pedra (Toca da Onça), Virtual Heritage, JAMstack Web Development.
\end{keywords}